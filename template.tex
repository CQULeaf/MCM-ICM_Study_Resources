分点描述:
一、实心圆点:
\textbf{加粗字体}
\begin{itemize}
	\setlength{\parsep}{0ex} 
	\setlength{\topsep}{2ex} 
	\setlength{\itemsep}{1ex} 
	\item 
	\item 
	\item 
\end{itemize}

二、前用序号引出要点:1. 2. 3.
\begin{enumerate}[\bfseries 1.]
	\setlength{\parsep}{0ex} %段落间距
	\setlength{\topsep}{0.5pt} %列表到上下文的垂直距离
	\setlength{\itemsep}{0.5pt} %条目间距
	\item 
	\item 
	\item 
\end{enumerate}

三、前用带括号的序号引出要点:1) 2) 3)
\begin{enumerate}[\bfseries 1)]
	\setlength{\parsep}{0ex} %段落间距
	\setlength{\topsep}{0.5pt} %列表到上下文的垂直距离
	\setlength{\itemsep}{0.5pt} %条目间距
	\item 
	\item 
	\item 
\end{enumerate}

三、其他的标记分点:\usepackage{amssymb}
\begin{itemize}
	\item[-] The major idea is...
	\item[$\diamondsuit$] The major idea is...
	\item[$\bigstar$] The major idea is...
	\item[$\blacklozenge$] The major idea is...
	\item[$\blacktriangleright$] The major idea is...
\end{itemize}

插入图片:如果需要图紧跟在文字下面,浮动体用[H]
\ref{pic}
\begin{figure}[htbp]  %h此处,t页顶,b页底,p独立一页,浮动体出现的位置
	\centering  %图表居中
	\includegraphics[width=.8\textwidth]{图片名称.格式} %图片的名称或者路径之中有空格会出问题  %.8用来调整图片大小
	\caption{图片标题}  
	\label{pic}
\end{figure}
\vspace{-0.5cm}


假设模版:Assumption 1:
\begin{enumerate}[\bfseries \textit{Assumption} 1:]
	\item \textbf{假设一}\\
	\textbf{\textit{Explanation:}} 解释
	\item \textbf{假设二}\\
	\textbf{\textit{Explanation:}} 解释
	\item \textbf{假设三}\\
	\textbf{\textit{Explanation:}} 解释
	\item \textbf{假设四}\\
	\textbf{\textit{Explanation:}} 解释
\end{enumerate}

三线表:Notations
\ref{tab}
\begin{table}[htbp]
	\begin{center}
		\caption{Notations}
		\begin{tabular}{m{1.5cm} m{12.1cm}}%具体数字依照具体情况更改
			\toprule[2pt]
			\multicolumn{1}{m{2.7cm}}{\centering Symbol}%具体数字依照具体情况更改
			&\multicolumn{1}{m{12cm}}{\centering Description }\\%具体数字依照具体情况更改
			\midrule
			$\mu$& The constant term \\
			\vspace{2pt}
			$\epsilon_t$& The error term \\
			\vspace{2pt}
			$\gamma_i$& Autocorrelation coefficient \\
			\vspace{2pt}
			$y_{t-i}$& Historical reported data
			\vspace{2pt}
		    $\theta_i$& Correlation coefficient
		    \vspace{2pt}
		    $\epsilon_{t-i}$& Historical reported data
		    \vspace{2pt}
		    $y_{t-i}$& Historical data error
		    \vspace{2pt}
		    $n$& Sample size
		    \vspace{2pt}
		    $L$& The great likelihood function
		    \vspace{2pt}
		    $K$& Number of model parameters
		    \vspace{2pt}
		    $t$& Number of tries
		    \vspace{2pt}
		    $p_i$& Percentage corresponding to the different number of tries
		    \vspace{2pt}
		    $\lambda_{max}$& The maximum eigenvalue of the judgment matrix
		    \vspace{2pt}
		    $n$& The order of the judgment matrix
		    \vspace{2pt}
		    $w_i$& Weights of neuron
		    \vspace{2pt}
		    $w_i$& Weighted sum of the input signals
		    \vspace{2pt}
		    $\theta_{(i)}$& Threshold
		    \vspace{2pt}
		    $f$& Activation function
		    \vspace{2pt}
		    $w_i$& Weights of neuron
		    \vspace{2pt}
		    $w_i$& Weights of neuron
		    
		    
			\\%字后一行下面没有vspace
			
			\bottomrule[2pt]
		\end{tabular}
		\label{tab}
		
		\begin{tablenotes}%以下需要notes才加
			\footnotesize
			\item[*] *There are some variables that are not listed here and will be discussed in detail in each section. %此处加入注释*信息
		\end{tablenotes}
	\end{center}
\end{table}

引用参考文献:
\cite{第几条}

编号的行间公式一个公式,一个号:
\begin{equation}
	公式
\end{equation}

编号的行间公式多个公式,一个号:
\begin{equation}
	\begin{aligned}
		& 公式一 \\
		& 公式二
	\end{aligned}
\end{equation}

编号的行间公式多个公式大括号括起来,一个号:
\begin{equation}
	\begin{cases}
		\begin{aligned}
			& 公式一 \\
			& 公式二  \\
			& 公式三
		\end{aligned}
	\end{cases}
\end{equation}

不编号的二级标题:
\subsubsection*{标题}

参考文献:MLA
\clearpage   %另起一页继续写。
\begin{thebibliography}{99}
	\bibitem{1} 
	\bibitem{2} 
	\bibitem{3} 
	\bibitem{4} 
	\bibitem{5} 
\end{thebibliography}

两张图片并排显示:
\usepackage{caption}
\usepackage{graphicx}
\usepackage{float} 
%\usepackage{subfigure}
\usepackage{subcaption}
\begin{figure}[htbp]
	\centering
	\begin{minipage}{0.49\linewidth}
		\centering
		\includegraphics[width=0.9\linewidth]{pic4}
		\caption{chutian1}
		\label{chutian1}%文中引用该图片代号
	\end{minipage}
	%\qquad
	\begin{minipage}{0.49\linewidth}
		\centering
		\includegraphics[width=0.9\linewidth]{pic5}
		\caption{chutian2}
		\label{chutian2}%文中引用该图片代号
	\end{minipage}
\end{figure}
