
\documentclass[withoutpreface,bwprint]{cumcmthesis} %去掉封面与编号页

%\documentclass{article}
%\usepackage{fontspec}
%\usepackage{xeCJK}

\usepackage{amssymb}
\usepackage{subfigure}	%用于排版多张图片
\usepackage{float}	%用于排版图片位置
\usepackage[backend=biber, style=numeric]{biblatex}
\addbibresource{reference.bib}	%引用样式,参考文献

\usepackage{url}
\title{全球去中心化数字金融市场的新时代策略}

\begin{document}

%\maketitle

\begin{abstract}

当前情况下,人们更加重视数字货币以便于便捷交易。因此,基于当前金融和经济形势的现实,我们的目标是提出一个全球去中心化的数字金融系统,并验证其在解决当前数字货币缺乏监管和匿名性问题上的可行性。我们将工作分为三个阶段。

首先,我们使用改进的DSGE模型来描述我们建立的系统。该模型涵盖了四个部门:家庭、公司、商业银行和中央银行。我们考虑了三种情况:国家完全放弃原有货币,不完全放弃原有货币,以及中央银行不发行数字货币。对于不同情况,我们都得到了经济稳定状态的金融和经济特征。我们的模型足够广泛,以适应不同国家的各种情况。我们继续研究数字货币技术冲击的宏观经济效应。

其次,我们选择了14个指标来衡量影响系统的关键因素,并将其分为四类,即接入因素、增长因素、稳定性因素和安全性因素。基于这些关键因素,我们提出了一个全球监管机制。此外,我们通过建立基于向量空间模型的KNN(k最近邻)分类器模型,专注于数字货币的洗钱风险,以帮助一个国家判断数字货币的洗钱风险。

最后,为了扩展我们的分析,我们修改了SAR模型以反映新金融系统的长期效应。我们选取了163个国家的经济自由指数,然后添加空间因素来研究中央银行发行数字货币的空间溢出效应。结果表明,新的货币系统模型的出现将逐渐提高银行业的完善度、全球经济的表现以及与每个国家的经济关系。

总而言之,我们模拟了不同数字货币比例的经济稳定状态约束。结果显示,当中央银行发行数字货币以完全替代原有货币时,可以避免经济的剧烈波动,并尽快达到稳定状态。



\end{abstract}

%目录
\tableofcontents

%新的一页
\newpage

\section{引言}

\subsection{背景}

信息技术的发展和数字货币价格的飙升有效地推动了数字货币的爆炸性增长。到目前为止,全球共有16,344种数字货币,总市值达到1202亿美元\cite{chen2018status}。作为一种新的货币形式,数字货币逐渐降低了交易成本并提高了交易效率。几乎每个国家的中央银行都在积极构建和发展合法的数字货币。
然而,这种货币存在的缺陷,如不稳定性、容易逃避监管机构的监管、容易被犯罪活动利用,不仅阻碍了数字货币的发展,而且加剧了世界金融系统的不稳定性。如何建立一个协调有效的全球数字金融市场体系及其相应的监管体系,是确保数字货币规范化和健康发展的首要任务

\subsection{我们的工作}

1. 我们首先构建了一个充分代表可行的全球去中心化数字金融市场体系的模型。

2. 接下来,我们找到影响金融体系在个体、国家和全球层面的接入、增长、安全和稳定的关键因素。

3. 此外,我们为金融体系建立了一套监管机制。

4. 最后,我们分析了金融体系对银行业、经济和国际关系的影响,然后预测它们的长期效应。

\begin{figure}[htbp]  %h此处,t页顶,b页底,p独立一页,浮动体出现的位置
	\centering  %图表居中
	\includegraphics[width=.9\textwidth]{微信图片_20231112161834.png} %图片的名称或者路径之中有空格会出问题 
	\caption{我们的工作} % 图片标题 
\end{figure}

\section{假设}

以下假设适用于本文中的所有模型:

1. 我们不要求每个国家发行统一的数字货币。

2. 我们不要求每个国家完全放弃传统货币并用数字货币替代它。

3. 数字货币只由中央银行发行。

\section{模型一:金融体系构建-DSGE模型}

在本文中,我们考虑建立一个全球分散的数字金融市场。我们不要求世界上每个国家发行统一的货币,也不要求国家完全放弃实物货币并用数字货币替代。我们希望建立的是一个由中央银行发行数字货币的金融系统。当然,这也是大多数经济体在现阶段讨论的可能性。更重要的是,我们只考虑最一般的情况,即数字货币只由中央银行发行。

本节的模型基于Barrdear和Kumhof(2016)\cite{barrdear2016macroeconomics}和钱勇(2018)\cite{qian2019economic}设计的模型。使用动态随机一般均衡模型(DSGE),通过探讨合法数字货币的引入对国家宏观经济效应的影响,可以分析中央银行引入数字货币的可行性。由于钱勇(2018)\cite{qian2019economic}假设中央银行的数字货币完全取代了实物现金,即国家完全放弃了原有货币,这与现实脱节,本节将改进DSGE模型,建立一个涵盖四个部门的模型——家庭、制造商、商业银行和中央银行。在考虑不同情况的基础上——国家完全放弃原有货币,不完全放弃原有货币和不发行数字货币,结合经济现实,我们分析了中央银行发行数字货币对不同国家以及对国家宏观经济的影响是否可行。

\subsection{假设}

1. 不考虑外币存款准备金,也不考虑进出口因素。

2. 数字货币和实物货币作为支付工具,也可以作为具有相同利率的计息资产。

3. 名义价格和名义工资的粘性。

4. 消费者的消费习惯在一段时间内不会改变。

5. 从银行和客户的角度看,所有银行存款都是无法区分的。

6. 理性人假设,即对个人来说,消费带来正面效应,工作带来负面效应。

7. 有两级供应商,最终供应商和中间供应商。中间制造商处于垄断竞争状态,最终制造商处于完全竞争状态。

8. 中间供应商完全依赖外部融资(银行存款)进行资本投资和雇佣工人。中间制造商的投资策略调整需要相应的成本。

9. 银行存款准备金等于中央银行数字货币的利率。银行存款准备金率是无风险利率。

10. 有一个利率走廊机制。商业银行有两个融资渠道,可以用于向公众融资或从中央银行借款。

11. 商业银行不保留货币资金。

12. 商业银行的利率定价方法是无风险利率加上商业银行的信用风险定价。银行存款利率低于商业银行常备贷款便利率。

13. 货币政策的目标是通货膨胀。


\subsection{变量命名}

\begin{table}[H]
	\centering
	\begin{tabular}{llll}
		\toprule[1.5pt]
		\textbf{缩写} & \textbf{描述} & \textbf{缩写} & \textbf{描述} \\
		\midrule[1pt]
		Pt & 名义商品价格指数 & $\mu$ & 中间产品需求弹性系数 \\
		Ct & 家庭实际消费 & At & 全要素生产率 \\
		Dt & 家庭持有的银行存款 & Y & 资本产出弹性系数 \\
		Bt & 中央银行数字货币的规模 & X & 科技水平变化 \\
		Et & 家庭持有的中央银行实物货币规模 & $\theta_t$ & 技术影响 \\
		Wt & 名义工资 & kt & 企业拥有的资本 \\
		Nt & 实际劳动供应 & It(Z) & 企业投资 \\
		$R_{D}^{-1}$ & 银行存款利率 & $R_{L}^{t}$ & 商业银行贷款利率 \\
		$R_{t-1}^{-1}$ & 中央银行数字货币利率及中央银行实际利率 & $B'_{t}$ & 商业银行贷款给中央银行 \\
		Gt & 企业利润分红 & Mt & 存款准备金 \\
		dt & 家庭持有的实际银行存款余额 & $R_{C}^{-1}$ & 存款准备金率 \\
		bt & 家庭持有的实际中央银行数字货币规模 & $R_{T}^{-1}$ & 中央银行贷款利率 \\
		et & 家庭持有的实际中央银行实物货币规模 & $\sigma$ & 商业银行在进行贷款业务中的成本 \\
		gt & 实际制造商利润分红 & Wt & 商业银行风险管理能力 \\
		Wt & 实物货币 & $\delta$ & 资本折旧率 \\
		ut & 效用函数 & $\rho_{W}$ & 商业银行风险管理能力平滑指数 \\
		$\phi$ & 工作负效用 & $\rho_{v}$ & 家庭持有货币平滑指数 \\
		$\phi_{d}$ & 使用银行存款的负效用 & $\rho_{r1}$ & 商业银行信用风险溢价平滑指数1 \\
		$\alpha$ & 家庭持有的银行存款比例 & $\rho_{r2}$ & 商业银行信用风险溢价平滑指数2 \\
		$\nu$ & 家庭持有货币比例 & $\rho$ & 通胀调整因子 \\
		$\nu_{t}$ & 家庭持有数字货币比例 & $\rho_{r1t}$ & 商业银行信用风险(有制度保障) \\
		$\beta$ & 时间间隔折现因子 & $\rho_{r2t}$ & 商业银行信用风险(无制度保障) \\
		Z & 制造商序列号 & $\phi_{r}$ & 生产者价格调整成本因子 \\
		$y_t(Z)$ & 中间产品 & $\phi_{k}$ & 企业资本调整成本因子 \\
		$y_t$ & 最终产品 & $R_{10}$ & 第0阶段商业银行信用风险(有制度保障) \\
		& & $R_{20}$ & 第0阶段商业银行信用风险(无制度保障) \\
		\bottomrule[1.5pt]
	\end{tabular}
	\caption{模型1的变量命名}
\end{table}




\subsection{模型}

{\heiti 家庭} 首先,结合实际情况,我们假设家庭的名义预算约束。
\begin{equation}
	P_t c_t+D_t+B_t+E_t=W_t n_t+D_{t-1} R_{t-1}^D+B_{t-1} R_{t-1}+E_{t-1} R_{t-1}+G_t
\end{equation}
把它转换成实际预算约束,即两边同时除以 $P_t$。
\begin{equation}
	c_t+d_t+b_t+e_t=w_t n_t+d_{t-1} R_{t-1}^D+b_{t-1} R_{t-1}+e_{t-1} R_{t-1}+g_t
\end{equation}
同时,我们假设家庭当前的效用函数。
\begin{equation}
	u_t=\log c_t-\phi n_t-\frac{\phi_d}{2}\left(d_t-d_{t-1}\right)^2
\end{equation}
其中,第一项代表实际消费给消费者带来的效用。参数$\phi$衡量工作的负效用,其值越大,负面效应越高。第三项描述了实际银行存款变化对家庭的负面效应。由于家庭使用银行存款时需要接受某些限制并需要支付一定的“成本”,银行存款的波动越大,对家庭的负面效应越大。

为了反映中央银行数字货币相对于银行存款的替代优势,我们引入第二个约束:
\begin{equation}
		alpha_t c_t \leq d_t 
\end{equation}
\begin{equation}
	alpha_t=1-v_t
\end{equation}
\begin{equation}
	ell v_t=\rho_v \ell v_{t-1}+\left(1-\rho_v\right)\left(\alpha+(1-\ell) v_{t-1}\right)+q_t
\end{equation}
这里,$v_t$ 表示家庭持有货币的比例,$\ell v_t$ 表示家庭持有数字货币的比例,$q_t$ 表示中央银行数字货币的影响,即当 $q_t$ 为正时,$q_t$ 的增加将增加家庭持有的数字货币量并减少银行存款的规模。

基于(2)和(4)最大化家庭的效用:
\begin{equation}
	\max E_t \sum_{j=0}^{\infty} \beta^j u_{t+j}
\end{equation}
$β$表示效用的相互作用折现因子,其值在 0 和 1 之间。通过解拉格朗日方程得到其一阶条件,在此不做过多解释。

{\heiti 公司} 根据新凯恩斯DSGE框架,假设有两家公司,最终公司和中间公司,最终公司处于不完全竞争。他们从中间公司$z$购买中间产品$y_t(z)$以生产最终产品$y_t$。$z$是制造商的连续编号,$z ∈ [0, 1]$。制定一个生产函数,我们得到:
\begin{equation}
	y_t=\left(\int_0^1 y_t(z)^{\frac{\mu-1}{\mu}} d z\right)^{\frac{\mu}{\mu-1}}
\end{equation}
为了解决最终公司利润最大化的一阶条件,你可以得到:
\begin{equation}
	y_t(z)=\left(\frac{P_t(z)}{P_t}\right)^{-\mu} y_t
\end{equation}
其中,总价格指数定义为:
\begin{equation}
	P_t=\left(\int_0^1 P_t(z)^{1-\mu} d z\right)^{\frac{1}{\mu-1}}
\end{equation}
假设所有中间公司的生产函数满足Cobb-Douglas生产函数,它们是:
\begin{equation}
	y_t(z)=A_t k_t^\gamma(z) n_t^{1-\gamma}(z)
\end{equation}
其中,At 代表总因子生产率,由整体经济的技术水平决定:
\begin{equation}
	A_t=e^{\left(\chi^t+\theta_t\right)} 
\end{equation}
\begin{equation}
	\theta_t=\rho_\theta \theta_t+\varepsilon_{\theta t}
\end{equation}
这里$χ > 0$表示技术水平随时间持续增长。$k_t$代表资本,相应的折旧率定义为$δ$,中间公司可以通过投资 $i_t(z)$来积累资本:
\begin{equation}
	k_{t+1}(z)=i_t(z)+(1-\delta) k_t(z)
\end{equation}
假设公司完全依靠银行贷款进行资本投资和支付工人工资,因此有:
\begin{equation}
	L_t(z)=W_t n_t(z)+P_t i_t(z)
\end{equation}
因此,相应的公司支付$R^L_tL_t(z)$,其中$R^L_t$是该期间的银行贷款利率。此外,价格调整和资本调整需要公司支付一定的管理成本。价格调整成本和资本调整成本定义如下:
\begin{equation}
	C_P\left(P_t(z)\right)=\frac{\phi_P}{2}\left[\frac{P_t(z)-P_{t-1}(z)}{P_{t-1}(z)}\right]^2 y_t(z) 
\end{equation}
\begin{equation}
	C_k\left(k_t(z)\right)=\frac{\phi_k}{2}\left[k_t(z)-k_{t-1}(z)\right]^2
\end{equation}
因此,中间公司的实际利润是:
\begin{equation}
	g_t(z)=\frac{P_{t(z)}}{P_t} y_t(z)-\frac{R_t^L L_t(z)}{P_{t-1}}-C_P\left(P_t(Z)\right)-C_k\left(k_t(Z)\right)
\end{equation}
中间公司的目标是最大化每个时期利润的预期折现值,可以表示为:
\begin{equation}
	\max E_t \sum_{j=1}^{\infty} \psi_{t+j, t+j-1} g_{t+j}
\end{equation}
同样,使用拉格朗日方程解决$n_t(z)$和$k_t(z)$可以得到(19)的一阶条件,这里也不详细解释。

{\heiti 商业银行} 商业银行有两个融资渠道,可以通过$D_t$的银行存款或向中央银行的$B′_t$贷款进行融资。筹集资金后,商业银行可以选择以存款准备金的形式在中央银行$M_t$存款,也可以将资金贷给公司$L_t$。因此,对于商业银行部门,有:
\begin{equation}
	L_t+M_t=D_t+B^{\prime}{ }_t
\end{equation}
将其转换为实际变量,然后表示为:
\begin{equation}
	l_t+m_t=d_t+b_t^{\prime}
\end{equation}
对于商业银行,实际利润是:
\begin{equation}
	h_t=\frac{m_{t-1} R_{t-1}^C}{\pi_t}+\frac{l_{t-1} R_{t-1}^L}{\pi_t}-\frac{d_{t-1} R_{t-1}^D}{\pi_t}-\frac{b_{t-1}^{\prime} R_{t-1}^T}{\pi_t}-\sigma l_t-\frac{w_t d_t}{m_t}
\end{equation}
在这些指标中,\( R_{t-1}^C \) 表示存款准备金率,而 \( R_{t-1}^T \) 表示向中央银行支付的利率,这两者构成了利率走廊的下限和上限。\( \sigma {lt} \) 代表商业银行开展贷款业务过程中的成本。\( \frac{d_t}{m_t} \) 表示存款准备金率,它衡量商业银行的信贷创造能力。\( w_t \) 反映了商业银行的风险管理能力。\( w_t \) 的数值越高,商业银行的风险管理能力就越低。参照钱易(2018)[3]的研究,我们假设 \( w_t \) 受到以下约束:
\begin{equation}
	w_t=\left(1-\rho_w\right) w+\rho_w w_{t-1}+j_t
\end{equation}
由于存款准备金和中央银行的数字货币都是中央银行的负债,我们在这里假设两者是相等的,即:
\begin{equation}
	R_t^C=R_t
\end{equation}
根据商业银行的利率定价方法,即无风险利率加上商业银行的信用风险溢价,有:
\begin{equation}
	R_t^D=R_t+r_{1 t} 
\end{equation}
\begin{equation}
	R_t^L=R_t+r_{1 t}+r_{2 t} 
\end{equation}
\begin{equation}
	r_{1 t}=\left(1-\rho_{r_1}\right) r_{10}+\rho_{r_1} r_{1 t-1}+\varepsilon_{r_{1 t}} 
\end{equation}
\begin{equation}
	r_{2 t}=\left(1-\rho_{r_2}\right) r_{20}+\rho_{r_2} r_{2 t-1}+\varepsilon_{r_{2 t}}
\end{equation}
最大化每个时期商业银行总折现利润的预期值:
\begin{equation}
	\max E_t \sum_{j=1}^{\infty} \psi_{t+j, t+j-1} h_{t+j}
\end{equation}
根据拉格朗日方程,得到最优一阶条件。

{\heiti 中央银行} 对于中央银行,需要维持其资产负债表的平衡:
\begin{equation}
	B_t+E_t+M_t=B_t^{\prime}
\end{equation}
中央银行发行数字货币,为中央银行创建了一个基于价格的货币政策工具。假设货币政策以通货膨胀为目标,$R_t$的设定受以下规则约束:
\begin{equation}
	R_t=(1-\rho)\left[\frac{1}{\beta}+\varphi_\pi\left(\pi_t-1\right)\right] \rho R_{t-1}+\varepsilon_t^R
\end{equation}
{\heiti 总平衡} 在上述四个部门的最优约束下,满足总平衡条件,即总产出等于总供给:
\begin{equation}
	y_t=c_t+i_t+C_P\left(P_t(Z)\right)+C_k\left(k_t(Z)\right)+\sigma l_t+\frac{w_t d_t}{m_t}
\end{equation}

\subsection{实施与结果}
该模型可应用于各种经济体。各国可根据其经济发展状况和国家综合发展水平确定各参数,得到稳态条件下的约束,并与国家当前状况比较,以确定发行数字货币是否可行。但为了简化,我们尽可能考虑所有国家的当前状况并合理确定参数。当前的LIBOR(美国)12月利率约为$3\%$,该利率用作基准利率,期间贴现因子$\beta$校准至0.971。世界上总工资与GDP的比例在发达国家和发展中国家之间差异很大。欧美国家可达到$50\%$以上,非洲国家普遍低于$20\%$。在某些国家,如中国,这一值显著较低。考虑到全球情况,将$\gamma$设为0.7。根据一年期银行存款利率与一年期银行贷款利率以及存款准备金之间的差异,将$r_10$和$r_20$设为$0.4\%$和$2\%$。其他参数的校准见下表。

%\begin{center}
%	\begin{tabular}{cc}
	%		\toprule[1.5pt]
	%		\makebox[0.3\textwidth][c]{符号}	&  \makebox[0.4\textwidth][c]{意义} \\
	%		\midrule[1pt]
	%		$ W $	    	& 某一小时内该路段运行总收益-总成本   \\ 
	%		$ W_0 $	    & 区分高峰和低峰的一个临界值  \\ 
	%		$ P $	    	& 线路在一小时内所有站的总上车人数 \\ 
	%		$ x $	    	& 线路在一小时内的车辆数 \\  
	%		$ T_t $	    & 长期趋势项 \\ 
	%		$ M_t $	    & 简单移动平均项 \\ 
	%		\bottomrule[1.5pt]

\begin{table}[h]
	\centering
	\caption{模型1的参数校准表}
	\label{table:parameter_calibration_model_1}
	\begin{tabular}{cccc}
		\toprule[1.5pt]
		参数 & 校准值 & 参数 & 校准值 \\
		\midrule[1pt]
		$\beta$ & 0.7 & $\rho_v$ & 0.8 \\
		$\phi$ & 1 & $\rho_w$ & 0.6 \\
		$\alpha$ & 0.5 & $\rho$ & 0.8 \\
		$\gamma$ & 0.7 & $\rho_{r1}$ & 0.8 \\
		$\chi$ & 0.02\% & $\rho_{r2}$ & 0.8 \\
		$\delta$ & 2\% & $\phi_{P}$ & 4 \\
		$w$ & 0.2 & $\phi_{k}$ & 4 \\
		$\sigma$ & 0.60\% & $r_{10}$ & 0.40\% \\
		$\mu$ & 5 & $r_{20}$ & 2\% \\
		\bottomrule[1.5pt]
	\end{tabular}
\end{table}

根据上述设定的参数值,计算稳态条件下的相应存款准备金比率和家庭消费占GDP的比例。当$\ell = 1$,即数字货币完全替代现有实物货币时,这一指标的稳态值分别为$17.8\%$和$75\%$。详见下表\label{table:the_result1s}。

\begin{table}[h]
	\centering
	\caption{结果表}
	\label{table:the_results1}
	\begin{tabular}{ccc}
		\toprule[1.5pt]
		值 & 存款准备金比率 (m/d) & 家庭消费占GDP的比重 (c/y) \\
		\midrule[1pt]
		$\ell = 1$ & 17.8\% & 75\% \\
		$\ell = 0.5$ & 26\% & 68.2\% \\
		$\ell = 0$ & 32.9\% & 63\% \\
		\bottomrule[1.5pt]
	\end{tabular}
\end{table}


近年来,一些主要经济体降低或取消了法定存款准备金比率,但像中国这样的一些国家保持了高法定存款准备金比率。目前,大多数国际储备保持约$15\%$的存款准备金比率。根据世界银行发布的数据,2017年全球最终消费在GDP中的比例约为$80\%$。本文假设不考虑外币存款和进出口。如果考虑这一因素,稳态指标应适当向下调整。总体而言,参数校准后的模型可以很好地匹配当前世界经济金融的特征。从表3中也可以看出,当$\ell = 1$时,模型最符合当前情况,这意味着在现阶段,当公众持有的全部货币为数字货币时,更有利于经济稳定。

因此,基于此模型,我们认为建立中央银行发行数字货币的金融系统对当前国际经济形势是可行的。

\subsection{未来讨论}
在上述模型中,我们假设中央银行发行数字货币的技术影响为$q_t$。在整个经济分析中,我们将其设置为正数。现在我们将其设置为不确定数。参考钱勇(2018)\cite{qian2018experimental}的经济设定,我们将中央银行数字货币$q_t$的标准差设为0.0006,并将$\ell$设为1。下图显示了对整体宏观经济的影响。

\begin{figure}[htbp]  %h此处,t页顶,b页底,p独立一页,浮动体出现的位置
	\centering  %图表居中
	\includegraphics[width=.9\textwidth]{微信图片_20231112161844.png} %图片的名称或者路径之中有空格会出问题 
	\caption{中央银行数字货币的技术冲击对消费和产出的影响} % 图片标题 
\end{figure}
\vspace{-0.8cm}

图2显示,当发生正面的数字货币冲击时,长期将有助于增加家庭消费和经济产出的增长,这进一步证明了模型的可行性。

\subsection{灵敏度分析}
我们让每个参数的校准值在原始校准值的$10\%$范围内波动。结果表明这不影响我们得出的结论,因此我们认为模型是稳健的。

\section{关键因素:主成分分析(PCA)}
由于模型1只讨论了中央银行实现发行数字货币的可行性,这对国家的经济增长和经济稳定具有重大意义,在本节中,我们分析了影响个体、国家和世界级别金融体系的因素。

我们选取了14个指标来衡量新金融体系接入、增长、稳定性和安全性的关键因素,并使用主成分分析(PCA)筛选出关键变量。最终,将其中的11个归纳为四大类,代表金融体系的接入、增长、稳定性和安全性的关键因素。

PCA简化了共线性变量,并仅用少数关键因素合成它们。它可以找到几个包含主要信息的线性组合,每个线性组合的信息之间没有共线性。所以我们有下面的原则:
\begin{equation}
	\left\{\begin{array}{c}
		Y_1=a_1 X=a_{11} X_1+a_{12} X_2+\cdots+a_{1 p} X_p \\
		Y_2=a_2 X=a_{21} X_1+a_{22} X_2+\cdots+a_{2 p} X_p \\
		\cdots \cdots \\
		Y_m=a_m X=a_{m 1} X_1+a_{m 2} X_2+\cdots+a_{m p} X_p
	\end{array}\right.
\end{equation}

\begin{itemize}
	\item $a_1 a_1^{\prime}=1$

	\item $Y_i, Y_j$ are irrelevant. $(i \neq j, \quad i, j=1,2, \ldots, m)$
\end{itemize}

基于这个问题,我们根据国际惯例,选择世界上人口超过2000万或GDP总额超过1000亿美元的国家。这些国家根据每个国家2017年的人均GDP分为八个等级。每个等级随机抽取相同数量国家的14个经济指标,得到一个14维随机向量$X = (X1, X2, ......, X14)$。经PCA计算,按照选定主成分的特征值$> 1$和累计贡献率$≥ 80\%$的原则,我们得到一个向量之间少量且不相关的线性组合集合:$Y1, Y2, Y3, Y4$。它们是原始变量度量$X1, X2, ·····, X14$的第一、第二、第三和第四主成分。
\begin{itemize}
	\item 接入因素:通货膨胀率、政府腐败指数和政府信用评级
	
	从主成分系数可以看出,第一主成分由通货膨胀率、政府腐败指数和政府信用评级组成。数字货币出现的原因之一是政府过度使用货币导致过度通货膨胀。比特币的发行是为了解决过度通货膨胀的问题,因此通货膨胀率的大小可以影响接入因素。政府的工作效率也影响数字货币接入因素,分为政府腐败指数和政府信用评级。政府越腐败,就越抵制数字货币——它将不受政府监管或因官僚主义导致高监管成本。更重要的是,如果政府信用高,信用基础相对坚强,可以支持信用货币的发行。
	
	\begin{table}[h]
		\centering
		\caption{标准差、方差贡献率和累积贡献率对应于标准化变量的前四个主成分}
		\label{tab4}
		\begin{tabular}{lcccc}
			\toprule[1.5pt]
			指标     & 主成分1 & 主成分2 & 主成分3 & 主成分4 \\ 
			\midrule[1pt]
			标准差      & 2.3651112 & 1.4535807 & 1.2823521 & 1.08245266 \\
			方差贡献率   & 0.3995537 & 0.1509212 & 0.1174591 & 0.13369315 \\
			累积贡献率   & 0.3995537 & 0.5504749 & 0.6679339 & 0.80162705 \\ 
			\bottomrule[1.5pt]
		\end{tabular}
	\end{table}
	
	\begin{table}[h]
		\centering
		\caption{对应于标准化变量前四个主成分的特征向量}
		\label{tab5}
		\begin{tabular}{lcccc}
			\toprule[1.5pt]
			影响因素 & \(Y_1\) & \(Y_2\) & \(Y_3\) & \(Y_4\) \\ 
			\midrule[1pt]
			国内生产总值增长率 & 0.206143 & 0.201552 & -0.38098 & 0.079807 \\
			利率 & 0.182811 & 0.11747 & -0.06887 & -0.04665 \\
			广义货币占总储备比率 & -0.21747 & 0.023287 & 0.026979 & 0.388422 \\
			广义货币(%GDP)& -0.14679 & 0.412974 & 0.038546 & -0.05428 \\
			通货膨胀率 & 0.276984 & 0.097393 & 0.223729 & -0.01165 \\
			政府腐败指数 & -0.37868 & -0.0355 & 0.126894 & -0.16746 \\
			政府信用评级 & 0.38939 & -0.08538 & -0.12208 & -0.12838 \\
			GDP & -0.21549 & 0.459926 & -0.23584 & -0.24366 \\
			经常账户 & -0.30186 & -0.24827 & 0.013424 & 0.285172 \\
			预算 & -0.03598 & -0.50062 & -0.32238 & -0.29023 \\
			人口 & 0.02716 & 0.346364 & 0.49569 & 0.047599 \\
			债务 & -0.14336 & 0.28264 & 0.308497 & 0.199055 \\
			汇率 & 0.057408 & -0.15336 & -0.23341 & -0.31339 \\
			失业率 & 0.215507 & 0.069393 & 0.459653 & 0.020225 \\ 
			\bottomrule[1.5pt]
		\end{tabular}
	\end{table}
	
\end{itemize}

\begin{itemize}
	\item 增长因素:广义货币(占GDP的百分比)、国内生产总值、预算
	
	从\ref{tab5}的主成分系数可以看出,第二主成分由广义货币(占GDP的百分比)、国内生产总值和预算组成。广义货币,M2(准货币),反映了社会的直接购买力和潜在购买力,可以用来衡量未来数字货币的增长趋势。国内生产总值是未来增长的前提和基础。结合预算水平,可以衡量未来增长势头。。
\end{itemize}

\begin{itemize}
	\item 增长因素:广义货币(占GDP的百分比)、国内生产总值、预算
	
	从\ref{tab5}的主成分系数可以看出,第二主成分由广义货币(占GDP的百分比)、国内生产总值和预算组成。广义货币,M2(准货币),反映了社会的直接购买力和潜在购买力,可以用来衡量未来数字货币的增长趋势。国内生产总值是未来增长的前提和基础。结合预算水平,可以衡量未来增长势头。。
\end{itemize}


\section{模型二:构建监管机制模型}


\subsection{整体设计}
由于数字货币作为支付和资产工具活跃在人们的交易和投资活动中,风险的存在不可避免地会在一定程度上损害交易者或投资者的利益。主要包括由价格波动引起的投机风险、洗钱等犯罪活动的非法使用风险、数字货币交易平台的非法商业风险以及资源浪费风险\cite{yang2018risks}。

对于数字货币的监管,本文认为有三种方式。一是分散监管资源,监管每一个发行方。在一定程度上,这可以解决数字货币的非法交易等问题。但如何利用所有有限的发行方来监管所有发行方?这需要国家依靠现有的计算机技术,使用现有的高端信息技术来监管现有的发行方。第二是以实物资产为支持的数字货币体系。这种实物资产应该被数字货币体系中的所有国家广泛接受。这需要统一全球主权国家的观点,并建立统一的数字货币体系规则,如如何支付、如何发行和如何清算。第三是由中央银行作为发行方发行法定数字货币。在主权国家信用支持下发行的数字货币,在发行上不具备去中心化的特征。交易方式仍然使用点对点交易。

本文采用第三种监管机制,即由中央银行发行法定数字货币。这也是世界各主权国家积极探索的方式。例如,中国、英国等国家正在研究设计法定数字货币的分配路径。中央银行发行数字货币可以避免货币价值波动对国家经济和金融系统的影响,并且可以快速传递货币政策,减少国内外的投机行为,同时在一定程度上打击非法和犯罪活动。为了实现这一理念,必须建立更完善的监管体系。

首先,我们需要确立数字货币的法律地位。中央银行发行的数字货币和中央银行的信用作为保证,数字货币具有一定的通用性,但顶层法律设计仍然需要考虑。所有国家需要制定相应的法律、法规和系统规范,根据国家的经济状况和金融体系建设的现状,结合数字货币的发展,确定数字货币的本质属性。让公众了解数字货币并学会使用数字货币,使数字货币逐渐取代现有的实物货币。此外,世界银行、国际货币基金组织等国际组织应建立相关的国际通用标准,为各国提供参考。

此外,我们需要建立监管机构。我们已确定数字货币的发行方为中央银行,因此在这个阶段最合适的监管机构仍然是中央银行。首先,因为大多数国家的中央银行已经建立了相关的研究部门,对数字货币有相对清晰的了解是可行的。其次,因为中央银行正承包发行数字货币,可以设定相应的发行策略。中央银行发行数字货币不能一蹴而就,需要逐步实现。否则,大量合法货币的涌入会导致整个金融体系的崩溃。最理想的方式是在发行法定数字货币的同时,撤销实物货币,并在实现数字货币替代实物货币的同时,保持整个金融体系内的货币供应稳定,确保国家有序健康地发展。总体而言,只有中央银行才能承担这一重任。

此外,我们需要建立账户实名制。中央银行可以选择将直接发行的货币发放给个人和企业,或选择通过商业银行将其发放给个人和企业。我们认为,根据国家的情况,各国可以根据当前的银行体系和实物货币的发行方式确定分配方法。

最后,我们希望在全球范围内构建统一的监管模型。一个国家的监管政策变化可能会导致该国的数字货币交易波动,从而影响其价值。由于数字货币的国际通用性,其他国家和持有其他国家数字货币的机构将面临风险。因此,各国政府应具备全球发展的理念,增强国际合作意识,探索数字货币监管方面的国际合作。充分发挥国际组织在全球统一监管体系中的重要作用,建立全球统一的监管框架,并敦促各国共享交易数据,实现数字货币更规范的发展。

尽管这样的监管体系可以更有效地防止数字货币在当前阶段的问题,但它并不能完全解释监管机制能够消除各种风险,如洗钱风险。因此,我们建立了以下模型来协助国家在发行法定数字货币后判断洗钱风险,从而完善监管机制。

\subsection{模型:基于向量空间的KNN(K最近邻)分类器模型}
本节使用HM Treasury构建的洗钱风险等级评估系统,获取一定数量的原始数据,并将这些原始数据作为训练集,应用基于向量空间模型的机器学习中的KNN(K最近邻)分类器模型,自动学习风险等级分类决策标准,并对需要评估的每种洗钱方法的风险等级进行分类。最终,建立一个监控全球数字货币的机制。这意味着当其他国家通过使用英国的洗钱风险评估系统积累了足够的原始数据后,他们可以使用我们的模型和自己的指标来确定各自地区的洗钱风险等级,而无需使用英国的风险评估系统。对于国家来说,不仅可以评估和监控其数字货币的洗钱风险,还可以节省长期使用英国风险评估系统的成本。

\subsubsection{假设}
1. 相邻假设:具有相同风险等级的洗钱方法构成一个类别。同一类型的洗钱方法将构成一个相邻区域,不同类型的相邻区域彼此不重叠。

2. 在国家金融体系中,每种洗钱方法是独立的,相互之间没有影响。

3. 在国家金融体系中,每种洗钱方法发生的概率相同。

\subsubsection{参数}


\subsubsection{变量命名}
\begin{table}[h]
	\centering
	\caption{模型2的变量命名规则}
	\label{tab6}
	\begin{tabular}{clcl}
		\toprule[1.5pt]
		缩写 & 定义 & 缩写 & 定义 \\ \midrule[1pt]
		\(i\) & 区域 \(i\) 的洗钱方法 & \(k_i\) & 执法机构对区域 \(i\) 通过的洗钱知识水平 \\
		\(T_{vsi}\) & 衡量区域 \(i\) 洗钱行为造成的损害程度的分数 & \(S_{ri}\) & 区域 \(i\) 的结构风险 \\
		\(a\) & 跨国转移资金的能力 & \(R_{wmg}\) & 有缓解等级的风险 \\
		\(M\) & 资金流动量 & \(\vec{v}_i\) & 区域 \(i\) 的洗钱方法向量 \\
		\(V\) & 资金流动速度 & \(k_j\) & kNN算法参数 \\
		\(Q\) & 待售商品数量 & \(j\) & 风险等级分类(高、中、低) \\
		\(P\) & 单位商品价格 & \(l_{size,i}\) & 区域 \(i\) 的合规水平 \\
		\(T_{lsi}\) & 衡量区域 \(i\) 发生洗钱事件时报告给执法机构的可能性的分数 & \(l_i\) & 区域 \(i\) 的大小 \\
		\(r_i\) & 区域 \(i\) 报告可疑活动给执法机构的可能性 & \(S_{ri}\) & 领域 \(i\) 的结构风险分数 \\ \bottomrule[1.5pt]
	\end{tabular}
\end{table}

\subsubsection{模型}
\begin{equation}
	T v s_i=f\left(a, M, I_i\right)
\end{equation}
\begin{equation}
	T I s_i=F\left(\text { size }_i, r_i, k_i\right)
\end{equation}
\begin{equation}
	S r_i=G\left(T v s_i, T I s_i\right)
\end{equation}
在公式(34)中,M也可以用${PQ}/V$代替。

由于不同国家的金融体系不同,不同的功能关系$(f,F,G)$将会出现。

\subsubsection{实施与结果}
向量空间模型将每种洗钱方法表示为由实数分量组合而成的向量$ vi = (Tvsi , TIsi , Sri , Rwmg)$。每个分量对应一个评估指标。我们从英国的洗钱风险评估系统获取了2015年十二个主题领域的原始数据。数字货币洗钱方法属于低风险类别。我们选择除数字货币领域之外的十一个领域作为训练集,将十一个四维向量降级为二维向量。然后我们将它们投影到2D平面上,并观察数据分布(图\ref{pic3})。

\begin{figure}[htfp]  %h此处,t页顶,b页底,p独立一页,浮动体出现的位置
	\centering  %图表居中
	\label{pic3}
	\includegraphics[width=.9\textwidth]{微信图片_20231112161917.png} %图片的名称或者路径之中有空格会出问题 
	\caption{数据可视化} % 图片标题 
\end{figure}

计算训练集中所有点与测试集中当前点之间的距离,并按距离递增顺序排序,选择距离当前测试集中的点最小的k个点,并将它们存储在数据结构Sk中。确定最初k个点所在类别中出现的频率$p_j$。返回第一个k个点中频率最高的类别作为当前点的预测分类。在判断空间中点与点之间的距离时,我们使用欧几里得距离。

\begin{figure}[htfp]  %h此处,t页顶,b页底,p独立一页,浮动体出现的位置
	\centering  %图表居中
	\label{pic4}
	\includegraphics[width=.9\textwidth]{微信图片_20231112161918.png} %图片的名称或者路径之中有空格会出问题 
	\caption{分类结果图} % 图片标题 
\end{figure}

图\ref{pic4}给出了$k=3$的一个例子。红线表示分类边界,三个类别分别用+、$\bigcirc$和*表示。我们可以发现,最靠近由“î”代表的数字货币洗钱方法的三个点属于低风险类别,因此数字货币洗钱方法属于每个类别的概率是P(高风险类别 | 数字货币洗钱方式)= 0,P(中等类别 | 数字货币洗钱方式)= 0,P(低风险类别 | 数字货币洗钱方式)= 1。分类器给出的结果是数字货币洗钱方式属于低风险类别,这与真实结果一致,表明我们的分类器是准确的。

我们还可以发现,英国2015年的数字货币洗钱方式风险水平较低,与赌场、高价值交易商和零售博彩的风险水平相同。他们的风险结构类似。因此,数字货币的监管可以借鉴赌场、高价值交易商和零售博彩的监管。假设我们已经获得了某一年数字货币洗钱方式的四个指标。我们仍然可以使用分类器判断该年数字货币洗钱方式的风险等级,并与前一时期相比较,得到数字货币洗钱风险的趋势,以便更好地进行监管。

\subsubsection{灵敏度分析}
1. 对k值的敏感性

kNN中的k值通常取决于分类问题本身的经验或知识。k通常取奇数以减少多个主要类别共存的可能性。常用的值是k=3和k=5,但k也取值在50到100之间较大的值,这也取决于训练集样本大小的大小。

2. 对训练集样本大小的敏感性

训练集的样本大小不断更新。我们可以判断需要评估的洗钱方式的风险等级,然后将其添加到训练集中。扩大下一次评估的训练集样本大小将更准确。

\section{模型三:未来效应-空间自回归模型}
在本节中,我们预测数字货币的引入对世界长期整体经济体系的未来发展的影响,包括对银行业的影响、对全球经济趋势的影响以及对国家间关系趋势的影响。由于经济效应在全球范围内相互联系并与其历史发展水平相关,我们预测模型的关键是解决国家之间的“传染性”经济联系的效应以及以前的经济水平对未来的影响。我们应用了空间自回归模型(SAR)来预测未来发展。

\subsection{假设}
1. 当前时期的国家行为往往受到其以前行为的影响(直接利益)。

2. 国家的当前表现往往受到其他邻近国家行为的影响,并将潜在地影响所有其他国家(间接利益)。

3. 通过简化为空间因素分析不可观测因素对因变量的影响。

4. 在计算国家与国家之间的距离时,使用国家的首都位置作为空间单位的计算点。

5. 我们完全信任《华尔街日报》和美国传统基金会发布的年度报告中的各种经济自由指数。此外,具有更多经济自由的国家或地区将比经济自由较少的国家拥有更高的长期经济增长和繁荣。

6. 国家之间的关系只考虑经济水平,因此简化为贸易关系。

7. 所有模型的误差项满足正态独立同分布

\subsection{变量命名}
基于世界经济自由指数年报和世界首都之间的距离,我们选择了163个数据齐全的国家进行空间自回归预测分析。

空间自回归模型是指在模型中设置因变量自相关(空间滞后因子)并引入空间权重矩阵,包括其他独立变量的模型形式。

当数字货币新加入国家货币体系时,必然会影响国家、地区和全球层面的银行系统效率和政府干预的独立性。任何国家的货币体系变化都会导致其邻国也会改变其货币体系的观念。从建模的角度来看,各国银行业未来发展的效应是来自其他国家的动态反馈效应,这反过来又会影响全球银行业的发展。同样,全球经济发展趋势和国际关系的发展趋势将随着数字货币的引入或数字货币在国家内部经济体系中的持续改进,动态地影响周围地区。

\begin{table}[h]
	\centering
	\caption{模型3的变量命名规则}
	\label{tab7}
	\begin{tabular}{cl}
		\toprule[1.5pt]
		缩写 & 描述 \\
		\midrule[1pt]
		\(B\) & 金融自由度(影响银行系统效率) \\
		\(E\) & 综合自由度(影响经济发展) \\
		\(T\) & 贸易自由度(影响国际贸易) \\
		\(\rho\) & 空间自相关系数 \\
		\(\beta\) & 独立系数 \\
		\(W\) & 163×163空间权重矩阵 \\
		\(\sigma\) & 随机干扰 \\
		\bottomrule[1.5pt]
	\end{tabular}
\end{table}

\subsubsection{模型}
建立模型的过程如下:

首先,我们为163个国家建立基于距离的空间权重矩阵:
\begin{equation}
	W=\left[\begin{array}{ccc}
		W_{1,1} & \cdots & W_{1,163} \\
		\vdots & \ddots & \vdots \\
		W_{163,1} & \cdots & W_{163,163}
	\end{array}\right]
\end{equation}
因此,我们得到了基于距离的初始空间权重矩阵W。计算$w_ij$后,通过行进行归一化处理得到了行归一化后的空间权重矩阵$W$。

考虑到我们之前的假设,我们需要引入滞后解释变量$B_t-1$, $E_t-1$,$T_t-1$,并建立以下改进的空间自回归模型:

模型A:预测未来银行业:
\begin{equation}
	B=\rho_1 W B+B_{t-1} \beta_1+\varepsilon_1
\end{equation}

模型B:预测各国未来经济发展:
\begin{equation}
	E=\rho_2 W E+E_{t-1} \beta_2+\varepsilon_2
\end{equation}

模型C:预测各国间未来贸易关系:
\begin{equation}
	T=\rho_3 W T+T_{t-1} \beta_3+\varepsilon_3
\end{equation}

基于模型对变量矩阵$B_t-1$,$E_t-1$,$T_t-1$的参数估计采取以下形式:
\begin{equation}
	\begin{aligned}
		& \hat{y}_1^{(1)}=\left(I_k-\hat{\rho}_1 W_1\right)^{-1} B_{t-1} \hat{\beta}_1 
	\end{aligned}
\end{equation}
\begin{equation}
	\begin{aligned}
		& \hat{y}_2^{(1)}=\left(I_k-\hat{\rho}_2 W_2\right)^{-1} E_{t-1} \hat{\beta}_2 
	\end{aligned}
\end{equation}		
\begin{equation}
	\begin{aligned}
		& \hat{y}_3^{(1)}=\left(I_k-\hat{\rho}_3 W_3\right)^{-1} T_{t-1} \hat{\beta}_3
	\end{aligned}
\end{equation}	

考虑到一个国家将数字货币作为官方货币引入对全球经济系统的影响,选择突尼斯作为基于世界上已经将数字货币确立为官方货币的国家的观察对象(2015年,突尼斯正式将数字货币作为官方货币。在接下来的几年中,突尼斯的经济自由指数显著增加。我们假设指数的增加或多或少来自于法定数字货币的发行)。将突尼斯的金融自由、总体自由和贸易自由增加$10\%$,得到$B′_t-1$,$E′_t-1$,$T′_t-1$,然后估算并分析全球影响:
\begin{equation}
	\begin{aligned}
		& \hat{y}_1^{(2)}=\left(I_k-\hat{\rho}_1 W_1\right)^{-1} B^{\prime}{ }_{t-1} \hat{\beta}_1 
	\end{aligned}
\end{equation}
\begin{equation}
	\begin{aligned}
		& \hat{y}_2^{(2)}=\left(I_k-\hat{\rho}_2 W_2\right)^{-1} E^{\prime}{ }_{t-1} \hat{\beta}_2 
	\end{aligned}
\end{equation}		
\begin{equation}
	\begin{aligned}
		& \hat{y}_3^{(2)}=\left(I_k-\hat{\rho}_3 W_3\right)^{-1} T^{\prime}{ }_{t-1} \hat{\beta}_3
	\end{aligned}
\end{equation}	

\subsubsection{实施与结果}
使用2017年和2018年的经济自由指数数据,利用MATLAB估算模型,并得到模型的系数如下:

\begin{table}[h]
	\centering
	\caption{模型估计系数}
	\label{tab8}
	\begin{tabular}{c|cccccc}
		\toprule[1.5pt]
		系数 & \(\hat{y}^{(1)}_1\) & \(\hat{y}^{(2)}_1\) & \(\hat{y}^{(1)}_2\) & \(\hat{y}^{(2)}_2\) & \(\hat{y}^{(1)}_3\) & \(\hat{y}^{(2)}_3\) \\ \midrule[1pt]
		\(\beta_i\) & 0.911459 & 0.917817 & 0.98891 & 0.989744 & 1.010008 & 1.010584 \\
		\(\rho_i\) & 0.08811 & 0.081985 & 0.022328 & 0.020279 & -0.00734 & -0.00848 \\
		\bottomrule[1.5pt]
	\end{tabular}
\end{table}

估算的$\hat{y}_1^{(2)}-\hat{y}_1^{(1)}, \hat{y}_2^{(2)}-\hat{y}_2^{(1)}, \hat{y}_3^{(2)}-\hat{y}_3^{(1)}$排除了突尼斯的数据(因为直接效应显著大于间接效应,我们这里只考虑间接效应),并将其排序到表\ref{tab8}中。

从图\ref{pic5}可以看出,在突尼斯使用数字货币作为官方货币后,大多数国家的指标都显示出增长(减法的结果大多大于0)。因此,数字货币作为官方货币后,对世界各国有一个间接的溢出效应。它对提升世界各国银行系统的效率、经济增长以及促进国家间贸易自由都有促进作用。

\begin{figure}[htbp]  %h此处,t页顶,b页底,p独立一页,浮动体出现的位置
	\centering  %图表居中
	\label{pic5}
	\includegraphics[width=.9\textwidth]{微信图片_202311121619181.png} %图片的名称或者路径之中有空格会出问题 
	\caption{消除直接效应数据对全球各国指标的影响} % 图片标题 
\end{figure}

\subsubsection{灵敏度分析}
考虑到数字货币的新兴水平对一个国家来说可能不会高达$10\%$,选择用$5\%$的增长再次测试。

图\ref{pic6}是$5\%$增长测试的结果。可以发现,尽管经济发展趋势有个别国家的出现和三个指标的增长不如以前明显。指标增长的趋势大致相同,总和大于零,这有助于世界经济的整体繁荣。

基于这个模型,我们可以得出结论,新的货币体系模型的出现将逐渐提高银行业的完善(包括采纳新兴数字货币)。此外,它可以通过增加经济自由度来增加GDP和国家经济活力,并增加国家之间商品和服务进出口的自由度和开放性。

\begin{figure}[htbp]  %h此处,t页顶,b页底,p独立一页,浮动体出现的位置
	\centering  %图表居中
	\label{pic6}
	\includegraphics[width=.9\textwidth]{微信图片_202311121619182.png} %图片的名称或者路径之中有空格会出问题 
	\caption{SAR模型的敏感性测试(移除直接效应数据后)} % 图片标题 
\end{figure}

\section{模型评估}

\subsection{优势}
\begin{itemize}
	\item 在模型1中,我们引入了数字货币持有比例,考虑了中央银行在发行数字货币时是否放弃原始货币。模型设计灵活,可根据需求随时调整。
	\item 在模型1中,我们考虑了家庭、制造商、商业银行和中央银行四个部门,可以系统地总结现实世界的发展,具有一定的实践意义。
	\item 模型2在理论上成熟,思维简单。它既可以用于分类也可以用于回归,并且可以用于非线性分类。训练时间复杂度低于支持向量机分类器。
	\item 在模型3中,我们创造性地提出使用空间度量知识来分析一个国家的中央银行发行数字货币对其他国家的空间溢出效应。基于地理因素研究中央银行发行数字货币对全球规模的影响具有实际意义。
	\item 在模型3中,基于数据可用性,我们考虑了163个国家,并从多个来源获取了大量数据,最终结果具有很高的可信度。
\end{itemize}

\subsection{弱点}
\begin{itemize}
	\item 在模型1中,尽管我们试图找到相关文献和相关数据以确保参数确定的合理性,但由于某些参数的值是主观的,可能会影响最终结果。
	\item 在模型1中,我们没有考虑外币存款,也没有考虑进出口因素对整体经济稳定性的影响,可能会导致结果不准确。
	\item 当模型2的样本大小较大时,计算量会太大,计算所需时间也会相应增长。
	\item 在模型3中,我们创造性地提出使用空间度量知识来分析一个国家的中央银行发行数字货币对其他国家的空间溢出效应。基于地理因素研究中央银行发行数字货币对全球规模的影响具有实际意义。
	\item 在模型3中,我们认为《华尔街日报》和美国传统基金会发布的年度报告中各国的经济自由指数是准确的,可以合理反映有关国家的实际情况。但这也可能在一定程度上影响整个模型的准确性。
\end{itemize}

\section{结论}
从我们提出的DSGE模型分析来看,建立全球去中心化的数字金融市场是可行的。我们提出了一个由中央银行发行数字货币的金融系统,并发现中央银行发行数字货币后,现有的经济和金融特征基本上满足了经济稳定状态,因此我们认为系统是可行的。此外,当公众持有所有货币作为数字货币时,即原始货币完全被放弃时,系统最可行。在后续研究中,我们发现当完全放弃原始货币时,正面的数字货币冲击可以实现国民经济的长期增长,支持系统的可行性。从我们最终建立的空间自回归模型来看,中央银行发行数字货币可以产生显著的正面空间溢出效应。从长远来看,中央银行发行数字货币对未来银行业发展、全球经济增长和国际关系有积极的影响。

\newpage

\section*{政策建议}
\addcontentsline{toc}{section}{政策建议}
\noindent 亲爱的国家领导人,

根据国际货币市场的要求,我们的团队制定了经验性和定量支持的模型,允许我们围绕新的数字金融市场提出各种政策建议。我们最关键的建议是建立一个由中央银行发行数字货币的全球去中心化数字金融系统。然后我们将给出我们建立的相应机制。我们的建议基于基于现实世界数据的精确建模和计算机模拟;因此我们对我们的提议充满信心。

首先,我们建议所有国家根据我们改进的DSGE模型建立一个中央银行发行数字货币的金融系统。毕竟,通过我们严格的分析,我们的模型是可行和通用的。在我们的分析中,当家庭持有的数字货币比例为1时,即当公众持有所有的钱作为数字货币时,它更有利于经济稳定。因此,我们强烈建议各国尽快发行数字货币,并使数字货币在每个国家更广泛地使用。

此外,我们建议领导人改进我们通过主成分分析识别的影响新兴金融系统接入、增长、稳定性和安全性的具体因素,包括通胀水平、政府腐败程度和政府信誉度;广义货币、国内生产总值、预算水平;人口、失业率;广义货币与总储备的比率、汇率等。

此外,我们建议各国根据我们建立的KNN分类器模型判断国家金融体系中数字货币领域的洗钱风险等级。国家领导人可以根据分类器的结果评估洗钱风险等级。与此同时,国家领导人还可以根据影响数字货币洗钱风险的具体因素改进,包括部门规模、部门内的合规水平、员工的专业性和技术水平,以及执法机构的力量等。

最后,我们建议首先发行数字货币的国家逐渐推动和鼓励其邻国发展数字货币。首先发行数字货币的国家逐渐推动和鼓励其邻国发展数字货币。使用我们的空间自回归模型分析发行数字货币国家对世界各国各种经济指标的影响,结果充分证明了数字货币的发现可以影响邻国甚至全世界的所有国家。空间距离越近,正面影响越大。因此,我们希望全世界所有国家改进的新金融体系可以形成“从点到面”,逐渐影响全球经济,并促进世界经济的共同繁荣。

为了更好地促进世界各地的经济发展,请考虑我们的政策建议。我们真诚地希望这些建议能帮助国家领导人建立一个更好的金融体系。
\\

此致,

1916704团队

\newpage






\appendix

\bibliography{reference}

\end{document}